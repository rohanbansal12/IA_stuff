% !TEX root = ../chem_ia.tex
\section{Conclusion}
Although the calculated value for the activation energy of the iodination of acetone reaction ($\SI{71.01\pm14.27}{\kilo\joule\per\mole}$) had a percent error of over $18\%$ as compared to the widely accepted literature value of $\SI{86.60}{\kilo\joule\per\mole}$, the trends determined through the course of the experiment and the evaluation of the potential limitations support the general process and seem to confirm the initial hypothesis and the empirical theory upon which it was proposed. Furthermore, the investigation clearly focused on analyzing the initial research question--\textbf{How does varying temperature affect the rate of reaction between iodine and acetone solution in the presence of a hydrochloric acid catalyst?}--as it allowed for the determination of the rate at multiple temperatures. The linear regression calculated through proper stastical methods also lends credibility on the determination of the activation energy and the result could have been improved through remediation of the most glaring errors described above.