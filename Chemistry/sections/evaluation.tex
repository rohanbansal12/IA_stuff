% !TEX root = ../chem_ia.tex
\section{Evaluation}

\subsection{Error Evaluation}

As with any laboratory experiment, it is crucial to both identify and evaluate errors related to equipment and experimental design. The large percentage error calculated above also emphasizes the need for a thorough analysis of potential sources of error throughout the experiment, and thus they are discussed below. It is obviously imperative to recognize that many of these errors are a result of the constraints of utilizing a school laboratory with limited access to equipment and controlled environments, however, finding potential room for improvement can still help in the description of obtained results and create further extensions for future work.

\subsubsection{Equipment}

\begin{table}[h!]
\centering

\begin{tabular}{|p{2.3cm}|p{1.5cm}|p{1.7cm}|p{1.5cm}|p{9.5cm}|}
\hline 
 Equipment & Absolute \newline Uncertainty & Smallest \newline Measurement & Percentage \newline Uncertainty & Significance/Improvement\\
 \hline
 Electric Timer (to determine time of reaction) & $\pm \SI{.05}{\second}$ & $\SI{37}{\second}$ & $.14\%$ & This is a very low percent error ($< .5\%$) and lends credibility to the results of the overall investigation. It is highly unlikely that human error with regards to the starting and stopping of the electric timer was a much larger contributor to overall issues with the final results, as opposed to the equipment itself. Thus, no material improvement is required for the purpose of time determinations. \\
 \hline
 $\SI{10}{\milli\liter}$ Graduated Cylinder (to measure out reactants) & $\pm \SI{.05}{\milli\liter}$ & $\SI{5}{\milli\liter}$ & $1\%$ & This is a moderate percent error, as it is still relatively low for the purpose and scope of the general investigation, but is quickly compounded. Because the volumes are used in a multitude of locations during the calculations (and there may be potential issues with the initial molarity), an improvement to the volumetric measuring process could drastically reduce overall experimental error. To achieve this beneficial result, a more precise measuring tool (such as a $\SI{1}{\milli\liter}$ syringe) could have been used.\\
 \hline
 Thermometer (to determine temperature of water baths) & $\pm \SI{.05}{\kelvin}$ & $\SI{287.71}{\kelvin}$ & $.01\%$ & This is an extremely low percent error ($< .1\%$) and strengthens the overall results, especially of the determination of the rate constant. The overall impact of this temperature fluctuation on the final result is very minimal and no equipment improvements are required for temperature determination. Controlling the temperature for the duration of the experiment was instead a more potent and potentially harmful issue for the results and study as a whole.\\
\hline
\end{tabular}
\caption{Equipment Error Evaluation}
\label{table:equipment_error_evaluation}
\end{table}

\newpage

\subsubsection{Experimental Design}

\begin{table}[h!]
\centering

\begin{tabular}{|p{3cm}|p{7cm}|p{7cm}|}
\hline 
 Cause/Description & Impact on Results & Potential Alleviation\\
 \hline
 Limited Temperature Range (with a potential outlier) & Due to time and some feasibility constraints, only $4$ unique temperatures were chosen at which to run trials for the final activation energy experiment. These temperatures were relatively close magnitude-wise, indicating limited issues with intermediate temperature trends, however extreme temperatures were neglected on both ends of the spectrum. This could have caused serious drifts in the gradient calculation through the linear regression process. Additionally, $1$ of the $4$ temperatures appeared to provide outlier data, which was not thoroughly investigated nor confirmed. &  Obviously, more temperatures could have been used for testing purposes to assess whether extremely high or low temperatures cause serious deviations in the rate of the overall reaction. This would have also assisted in limiting the uncertainty with regards to the linear regression. It is, however, important to consider that at these extreme temperatures the reactant mixtures (containing water) could begin to boil or freeze, further introducing a confounding variable into the experiment.\\
 \hline
 Determination of completion of reaction (Random) & As iodine provides a strong color to the reactant mixture, a lack of color was relied for determining when the reaction had finished. This leads to potential variance with regards to the "specific" instance at which the reaction was truly finished and limits the consistency between trials. This issue is especially pertinent for quick trials (such as those at high temperatures), as the change is relatively rapid, and any hesitations or lack of sureness can have a serious impact on the final results. &  To remedy this, a spectrophotometer can be used to accurately measure wavelength absorbance of the reactant mixture during the duration of the reaction. This would ensure a consistent reading and limit human error with regards to the determination of completion. Due to issues with maintaining temperature equilibrium and the lack of resources, the device was not used for this investigation.\\
 \hline
 Inadequate Temperature Control (Random) & The impact of temperature on the rate constant inherently relies on maintaining a constant temperature for all reactants for the entirety of the reaction. However, because of the need for manual swirling while the reactants were in the water bath in addition to the open environment of the laboratory, it was difficult to accurately maintain the isolated temperature consistently. &  To ease the swirling process, an external magnetic stirring device could have been utilized to provide both more uniform mixing and limit the time spent outside the bath by the reactants. Ideally, a closed environment could be used to minimize external temperature fluctuations, however this was outside of the potential scope of this investigation. \\
 \hline
  Unaccounted change in concentrations of acid and acetone (Systematic) & The rate law determined for the reaction is $2nd$ order and depends on the concentration of $HCl$ and $Acetone$. To determine the rate, the assumption was made that the concentrations of these two reactants/catalysts remained constant, but there were obviously minor reductions for the concentration of both. This means that the simplification used to calculate the rate was not entirely accurate, as the rate did not stay even for the duration of the reaction. &  Some more complex mathematical techniques have been used in previous literature to attempt to better account for variations in the rate of the reaction over time, but they are largely out of the scope of this investigation. Overall, it has been found that ensuring $[HCl]$ and $[Acetone]$ significantly outweigh $[I_2]$ can minimize the error introduced by this issue, so it probably had limited impact on the final results.\\
 \hline

\end{tabular}
\caption{Experimental Design Error Evaluation}
\label{table:experimental_design_error_evaluation}
\end{table}

\subsubsection{Summary}

As demonstrated by \cref{table:equipment_error_evaluation}, there were limited equipment based errors for the majority of this experiment. Both of the primary tools used for measurement had uncertainties less than $1\%$, indicating that higher precision would not have drastically improved confidence or accuracy of the final results. It is important to note that the use of this equipment led to random error, as the determination of when to start and stop the timer was subject to the experimenter and varied between trials. \cref{table:experimental_design_error_evaluation} highlights both systematic and random errors associated with the actual design of the experiment itself. The lack of temperatures at which trials were conducted is the first major example of a systematic error, as this contributed to the final linear regression and drastically increased the uncertainty of the coefficients. Because the statistical approach of confidence intervals is inherently predicated on the degrees of freedom in the experiment (which approximately relates to the total data points), a lack of temperature ranges was a major inhibiter in producing more definite results. The lack of accountability in the change in concentration of the non-iodine reactants also provided another source of systematic error that could have been remedied to more accurately determine the rate of reaction. Prior literature has suggested that the concentrations of these reactants both undergo somewhat material fluctuations and can have large impacts on the rate of the otherwise slow iodination of acetone. The random errors of inadequate temperature control and a subjective mechanism for determining the completion of the reaction were also present, but would have required more complicated equipment and procedures to effectively control, which suggests that their alleviation is outside of the scope of this investigation. Overall, the investigation had a multitude of strengths and weaknesses in both the equipment utilized and the experimental procedure designed, which were partially discussed above and will be further analyzed in the subsequent comparison to documented scientific literature.

\subsection{Comparison to Scientific Literature}

The percent error of the result generated through this experiment ($\SI{71.01\pm14.27}{\kilo\joule\per\mole}$) and the accepted literature value from multiple papers ($\SI{86.60}{\kilo\joule\per\mole}$) was calculated to be $18\%$ in the results section. This error is larger than the $12\%$ uncertainty of the calculated result, indicating a serious systematic error in the methodology of this experiment. Therefore, it is worth comparing the procedure detailed throughout this paper and those employed by the various literature to analyze potential areas for improvement and justifications for the variation in the final result. The first approach was detailed by \textcite{other_literature_1} in which small quantities of the reaction mixture were drawn from the total reaction and passed through sodium bicarbonate and thiosulfate solution. The large concentrations of $HCl$ and $Acetone$ (as compared to $[I_2]$) was similar to that employed here, but the repeated usage of sodium bicarbonate and thiosulfate solution and the subsequent liberation of iodine was used to determine the rate of disappearance of iodine (or the overall rate of reaction of the iodination of acetone). This method allowed for more frequent assessments of the rate of the iodination reaction and provided more datapoints to ensure that the change in concentration of the non-iodine reactants was not affecting the rate. However, repeatedly opening the reactant solution to draw these samples out may have caused a similar issue to the presented investigation in which the temperature fluctuated. Overall, this procedure was unique in its lack of reliance on the light-absorbance of iodine solution, and was thus able to produce results with very limited uncertainty. In fact, the focus of the investigation was to minimize potential errors in the apparatus. Much of the difficulty in thoroughly comparing this investigation to \textcite{other_literature_1} is the large divide in equipment access and time constraints.

\textcite{reversiblity} also proposed a very similar experiment to \textcite{other_literature_1} by using titration techniques to repeatedly determine the rate of removal of $I_2$ from the reactant mixture, however they incorporated a buffer solution via the addition of potassium iodate to maintain the relative pH value of the overall solution. This helped introduce the discovery that this reaction between iodine and acetone is highly reversible in strongly acidic solutions. It is important to recognize that $HCl$ simply behaves as a catalyst in this reaction, it is not an active reactant, and thus it lowers the activation energy of both the forward and backward reaction. Both of the reactions are exceedingly slow in the absence of a catalyst (especially as compared to the general halogenation of many organic compounds), however the introduction of an acid catalyst allows for a more feasible reaction in both directions. At moderately acidic pH values for the reaction mixture, this allows for an eventual equilibrium shifted distinctively to the right. However, for more concentrated acidic solutions ($>\SI{.05}{\molar}$), the backwards reaction becomes more feasible, shifting the equilibrium towards the left and preventing the complete halogenation of the reaction. This result is particularly influential for the investigation presented here, as the production of reactants in the reverse reaction would have artifically inflated the total duration of the reaction and produced incorrect calculations for the rate of consumption of $I_2$ solution. The introducation of buffer solutions, as used by \textcite{reversiblity}, can prevent large fluctuations in the pH of the solution mixture, especially after the intermediate production of enols (which are slightly acidic themselves). This result also indicates an "optimal" concentration of $H^+$ ions at which the equilibrium constant is maximum and the time required for equilibrium to be reached is minimum, which would be worth further investigation via the use of various acidic concentrations.

A more traditional approach to the study of this reaction is also described by \textcite{main_literature}, where the absorbance of iodine and the change in absorbance of the reaction mixture is measured with the help of a spectrophotometer to determine the rate of the reaction at various points. This procedure was able to generate highly consistent (and accurate) results with limited trials through the use of a differential method for activation energy calculations. Although the mathematical premise of the investigation falls outside of the scope of this research, the introduction of absorbance data could facilitate an easier mechanism for calculating the activation energy and provide a more relevant baseline for comparison with the linear regression. The use of a spectrophotomer does require constant transfer of small quantities of the reaction mixture from the temperature-stable vessel to the external machine. To minimize heat loss to the surroundings, \textcite{main_literature} were able to use thermostatted pipettes which would most likely not have been feasible for this experiment.

\subsection{Extensions}

Many of the potential methodological extensions/improvements to this investigation were discussed in the comparison to external literature, however, there are other relevant extensions that could provide important information for industrial and theoretical purposes. Com{}paring various acidic catalysts and alternative halogens for this type of reaction could assist in determining optimal techniques for mass production of enols and improving the ability to rationalize the differences in proposed mechanisms for enolization in acidic-solutions. It would also be intriguing to compare this investigation to the same reaction with a basic catalyst. These experiments could provide an ability to investigate different potential products and the direct impact of the pH of the reactant mixture on the rate and mechanism of the general iodination of acetone~\parencite{base_catalyst}.