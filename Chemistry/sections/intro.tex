% !TEX root = ../chem_ia.tex
\section{Introduction}
When learning about catalysis, I was particularly intrigued by the specific subset of acid catalysts. Acid catalysts are often regarded as the most important area of catalysis in industry. More specifically, these acids allow for the protonation of a double bonded oxygen group in various carbonyls, including ketones such as acetone~\parencite{acid_catalyst}. As acetone is a major organic solvent which sees its own applications in industry and personal life, this combination of acid catalysis and organic chemistry is particularly relevant. A common iteration of this mechanism results from the use of iodine (a frequently found halogen in organic chemistry) due to its strong color~\parencite{iodine_coloring}. Thus, I was drawn to investigating the reaction between acetone and iodine in the presence of a hydrochloric acid catalyst and determining the activation energy ($E_a$) by looking for the loss of color in the reaction solution.

\subsection{Research Question}
\textbf{How does varying temperature affect the rate of reaction between iodine and acetone solution in the presence of a hydrochloric acid catalyst?}

This question will be answered through empirical theory and experimentation, and the resulting data will also facilitate the calculation of the activation energy of this general reaction.
