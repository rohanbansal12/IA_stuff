% !TEX root = ../chem_ia.tex
\section{Variables}

\subsection{Rate Law Experiment}

\textbf{Independent Variable: Concentration of reactants ($\frac{mol}{dm^3}$)} 

It has empirically been confirmed that the rate of a reaction is solely dependent on the rate constant (which is constant at a fixed temperature) and the concentrations of the reactants (and catalysts in the case of specific mechanisms). Thus, to find the rate law for this specific iodination of acetone reaction, a baseline was provided through the initial configuration and then each set of trials manipulated the concentration of a certain reactant ($HCl$, $Acetone$, or $I_2$) for the purpose of comparison. This allowed for the determination of the order of each reactant in the rate law through simple proportions. To make this process easy, the baseline configuration used $5$ $mL$ of each reactant/catalyst, and the manipulated trials doubled these volumes for each reactant to $10$ $mL$. The total volume was held constant through the addition of water, so this doubling in volume of the reactant meant a direct doubling of that reactant's concentration, and thus the effect on the rate could easily be identified. As the rate law is crucial to the determination of the activation energy in the second part of this investigation, the choice of independent variable is justified.

\textbf{Dependent Variable: Time taken for solution to lose its color ($s$)} 

By using the time for the completion of the reaction (once the iodine color has disappeared) and the initial concentration of iodine, the rate of the reaction can be determined for each chosen configuration of reactants. These rates can then be compared to the baseline provided by the initial configuration for the experiment to identify how the changing of concentrations of each reactant impacts the overall rate. By doing this for each reactant, the rate law for the total reaction can be empirically confirmed/found, allowing us to determine the rate constant in the activation energy experiment. Additionally, as time is also the dependent variable for the activation energy experiment, the trials for the first configuration can then be utilized further. Thus, the time taken for the reaction to lose its color is relevant to the overall investigation.

\textbf{Unique Control Variable: Temperature ($K$)} 

Because the temperature impacts the rate constant ($k$), it was crucial to maintain a constant temperature for the determination of the rate formula. As the premise of finding the rate law relies on the resultant rate proportions of changing concentrations, a change in $k$ would introduce a confounding variable and render the results inconclusive. A room temperature of $294.8$ $K$ was taken as this constant temperature for all trials conducted as part of the rate law experiment and it was ensured that all reactants were at this fixed temperature prior to the initialization of the reaction through constant temperature checks via thermometer.

\subsection{Activation Energy Experiment}

\textbf{Independent Variable: Temperature ($K$)} 

As mentioned in the background, the rate constant varies depending on temperature, and the calculation of multiple rate constants at known temperatures can be used to find the activation energy through the Arrhenius equation. The linear relationship between the inverse of the temperature and the natural logarithm of the rate constant allows for simple graphing and gradient calculation which can then be used to find the activation energy, indicating the relevance to the investigation. Because specific temperatures are not required and instead a variety of temperatures are crucial, temperatures in the ranges of $283-285$, $293-298$ (room temperature in the above experiment), $303-308$, and $313-318$ $K$ were used to conduct trials and were manipulated through the use of heated or ice-filled water baths. I refrained from adding higher temperatures for fear of partial evaporation of the reactants, as this would impact concentrations, and lead to erroneous calculations with regards to the rate of reaction and rate constant.

\textbf{Dependent Variable: Time taken for solution to lose its color ($s$)} 

This is an easily observable characteristic of the reaction which allows for the determination of the completion of the reaction and thus leads to a straightforward process for quantifying the rate of the reaction (as discussed previously). All times were calculated in seconds ($s$) for consistency and to maintain the appropriate rates for $k$ for a second order reaction ($\frac{dm^3}{mol \cdot s}$). Because the rate constant can then be used to determine the activation energy of the reaction, this choice of a dependent variable is entirely relevant to the investigation and allows for the answering of the initial research question. 

\textbf{Unique Control Variable: Volumes of reactants ($cm^3$)} 

All trials utilized for this experiment had identical volumes of each reactant and identical total volumes leading to the same concentration of each reactant for all trials. These volumes were chosen to make iodine the limiting reactant and provide large excess of $HCl$ and $Acetone$, so as to ensure that all iodine was used up after the reactions completion and prevent large fluctuations in the concentration of the other reactants which could also impact the rate. This was necessary as the experiment was designed to investigate the change of the rate due to temperature alone, and thus the control variable was necessary.

\subsection{Universal Controls}

\textbf{Pressure}

Large fluctuations in pressure can impact the rate of evaporation of the reactants involved in the experiment which would affect the concentrations and induce erroneous rate calculations. Pressure changes have also been found to impact the kinetics of molecules with regards to aqueous reactants which could have introducing confounding causes for the change in the rate constant with varying temperatures. To control for pressure, all experiments were carried out in a controlled environment and a uniform laboratory at a fixed altitude. 

\textbf{Purity/Molarity of Reactants}

As the molarity of the reactants was used to determine their respective concentrations, it was crucial that these molarities were accurately measured and maintained for the duration of all trials. To control, all reactants were properly diluted prior to the beginning of the experiment and sufficient quantities were produced so as to allow for consistent utilization for all trials. The diluted solutions were stored safely to prevent any external contamination or evaporation.

\textbf{Physical Manipulation of Reaction}

Swirling of the reaction mixture is crucial for ensuring that the reactants evenly mix throughout and helps faciliate more frequent collisisons between particles (thereby increasing the rate of the reaction). To ensure that discrepancies in swirling did not account for the various results, all trials were given $30$ seconds of swiriling and then allowed to sit to completion. This attempted to control the confounding variable and let the reaction occur naturally after all reactants were adequately mixed.
