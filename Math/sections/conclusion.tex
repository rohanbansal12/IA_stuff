\section{Conclusion}

This investigation attempted to explore the fascinating world of physical moment of inertias by highlighting unique methods of mathematical derivations. Using the basic definition of moment of inertia presented at the beginning of the paper, it was trivial to determine the value for simple point masses or hollow shapes. Basic integration, elemental mass areas, and vectorized/matrix-oriented approaches were all utilized for further calculations for more complex bodies, culminating in the final extension whic provides a result to calculate the moment of inertia for any 3-D body about any axis, obviously an extremely powerful and streamlined tool for physics-based math. 

Scientists and mathematicians continue to work on deriving methods for these types of calculations for highly irregular bodies, such as those found in nature, and on optimizing rotational inertia to improve engineering-based applications in the real-world. Advancements in multivariable calculus and matrix factorization have already allowed for more complex implementations of the extension formula