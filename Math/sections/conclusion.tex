% !TEX root = ../math_ia.tex
\section{Conclusion}

This investigation attempted to explore the fascinating world of physical moment of inertias by highlighting unique methods of mathematical derivations. Using the basic definition of moment of inertia presented at the beginning of the paper, it was trivial to determine the value for simple point masses or hollow shapes. Basic integration, elemental mass areas, and vectorized/matrix-oriented approaches were all utilized for further calculations for more complex bodies, culminating in the final extension which provides a result to calculate the moment of inertia for any 3-D body about any axis, obviously an extremely powerful and streamlined tool for physics-based math. 

Scientists and mathematicians continue to work on deriving methods for these types of calculations for highly irregular bodies, such as those found in nature, and on optimizing rotational inertia to improve engineering-based applications in the real-world. Advancements in multivariable calculus and matrix factorization have already allowed for more complex implementations of the extension formula, and the results emphasize the importance of mathematics in the physical realm. Obviously, a logical step for further research and the verification of the results presented here would be a real-world study of the movement of these uniquely shaped bodies. By providing a known angular velocity, the theoretical results could be confirmed or rejected, and potentially allow for more nuanced discoveries about internal and environmental factors which affect the rotational movement of rigid bodies. It is obviously imperative to remember that these are all purely abstract derivations and thus simply provide our approximate and ideal guess as to the moments for the bodies discussed previously. The deliberate mathematical approach is both a strength and weakness as it limits external factors and avoids the issue of confounding and uncontrollable factors which may plague a practical laboratory experiment, but also may struggle to account for individual discrepancies or allow for a more nuanced debate surrounding external factors that contribute to the findings. Overall, the work still facilitates a general understanding of moment of inertia for progressively more complex bodies while highlighting the future of the field and discussing new techniques which can improve the generalizablity of these important results.