% !TEX root = ../math_ia.tex
\section{Introduction}
\label{sec:introduction}
I first became intrigued with the concept of moment of inertia when learning about rotating bodies such as flywheels and ferris wheels. I understood that these bodies were "moving" and stored kinetic energy, but they had no linear movement, leaving me somewhat confused about the general premise for calculating their stored energy. I soon realized that an understanding of rotational movement and a particle based approach could higlight the parallels between linear kinetic energy and rotational kinetic energy which led to the discovery of moments of inertias. The literal definition of moment of inertia is the opposition of a body to having its speed of rotation abbout an axis altered by the application of a torque (the rotational equivalent of traditional inertia). It plays a major role in rotational dynamics and has a host of applications when it comes to motors, wheels, and machinery. Its ability to serve as a measure of the resistance to angular motion for a body makes it equally important as traditional inertia, which dictates many physical aspects of our daily lives. Pure rolling motion, bridge building, and flywheels in engines are all predicated on the concept of moment of inertia, and thus the idea becomes exceedingly crucial in the study and implementations of engineering. Scientists and mathematicians are continuing to look for methods to calculate more intricate moments of inertias to improve current physical systems and create new practical technology~\parencite{Young_Freedman_Young_2020}.


This interesting concept lends itself to a a mathematical investigation for deriving moment of inertia formulas for progressively more complex mass distributions. Because of the reliance of moment of inertia on the geometric makeup of a body, I decided to investigate the various approaches that are used to calculate the values and derive some of the most commonly used formulas in modern physics and mathematics.